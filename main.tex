%% start of file `template.tex'.
%% Copyright 2006-2013 Xavier Danaux (xdanaux@gmail.com).
%
% This work may be distributed and/or modified under the
% conditions of the LaTeX Project Public License version 1.3c,
% available at http://www.latex-project.org/lppl/.


\documentclass[11pt,a4paper,sans]{moderncv}        % possible options include font size ('10pt', '11pt' and '12pt'), paper size ('a4paper', 'letterpaper', 'a5paper', 'legalpaper', 'executivepaper' and 'landscape') and font family ('sans' and 'roman')

% moderncv themes
\moderncvstyle{classic}                             % style options are 'casual' (default), 'classic', 'oldstyle' and 'banking'
\moderncvcolor{green}                               % color options 'blue' (default), 'orange', 'green', 'red', 'purple', 'grey' and 'black'
%\renewcommand{\familydefault}{\sfdefault}         % to set the default font; use '\sfdefault' for the default sans serif font, '\rmdefault' for the default roman one, or any tex font name
\nopagenumbers{}                                  % uncomment to suppress automatic page numbering for CVs longer than one page

% character encoding
\usepackage[utf8]{inputenc}                       % if you are not using xelatex ou lualatex, replace by the encoding you are using
%\usepackage{CJKutf8}                              % if you need to use CJK to typeset your resume in Chinese, Japanese or Korean
\usepackage[document]{ragged2e}

% adjust the page margins
\usepackage[scale=0.75]{geometry}
%\setlength{\hintscolumnwidth}{3cm}                % if you want to change the width of the column with the dates
%\setlength{\makecvtitlenamewidth}{10cm}           % for the 'classic' style, if you want to force the width allocated to your name and avoid line breaks. be careful though, the length is normally calculated to avoid any overlap with your personal info; use this at your own typographical risks...

% personal data
\name{Louis}{Primeau}
%\title{Resumé title}                               % optional, remove / comment the line if not wanted
%\address{678 Bathurst St}{M5S 2R3 Toronto}{Canada}% optional, remove / comment the line if not wanted; the "postcode city" and and "country" arguments can be omitted or provided empty
\phone[mobile]{+1~(949)~572~7975}                   % optional, remove / comment the line if not wanted
%\phone[fixed]{+2~(345)~678~901}                    % optional, remove / comment the line if not wanted
%\phone[fax]{+3~(456)~789~012}                      % optional, remove / comment the line if not wanted
\email{louis.primeau@mail.utoronto.ca}                               % optional, remove / comment the line if not wanted
%\homepage{www.johndoe.com}                         % optional, remove / comment the line if not wanted
%\extrainfo{additional information}                 % optional, remove / comment the line if not wanted
%\photo[64pt][0.4pt]{picture}                       % optional, remove / comment the line if not wanted; '64pt' is the height the picture must be resized to, 0.4pt is the thickness of the frame around it (put it to 0pt for no frame) and 'picture' is the name of the picture file
%\quote{Some quote}                                 % optional, remove / comment the line if not wanted

% to show numerical labels in the bibliography (default is to show no labels); only useful if you make citations in your resume
%\makeatletter
%\renewcommand*{\bibliographyitemlabel}{\@biblabel{\arabic{enumiv}}}
%\makeatother
%\renewcommand*{\bibliographyitemlabel}{[\arabic{enumiv}]}% CONSIDER REPLACING THE ABOVE BY THIS

% bibliography with mutiple entries
%\usepackage{multibib}
%\newcites{book,misc}{{Books},{Others}}
%----------------------------------------------------------------------------------
%            content
%----------------------------------------------------------------------------------
\begin{document}
%-----       resume       ---------------------------------------------------------
\makecvtitle
\section{Nationality}
\cvitem{Citizenship:}{US, Canada}

\section{Education}
\cventry{2018--}{Undergraduate}{University of
  Toronto}{Toronto, ON CA}{\textit{-}}{Major: Engineering Science,
  cGPA: 3.94}
\cventry{2014--2018}{High School}{Orange County School of the
  Arts}{Santa Ana, CA USA}{\textit{-}}{Creative Writing Conservatory}

\section{Work Experience}
\cvitem{2020}{USRA Recipient, Intelligent
  Microsensory Systems Lab under Professor Roman Genov. Working on the
  implementation and behavioral simulation of neural differential equations over
  a memristive crossbar.}
\cvitem{2019--}{Undergraduate Research Assistant, Intelligent
  Microsensory Systems Lab. The project goal is to create a memristive
  crossbar, an analog matrix multiplication chip. Worked on the
  interface to the crossbar and in-hardware processing, involving coding in Verilog and C for Microblaze.}
\cvitem{2019}{Data Science Intern, EATLAB, a startup working in
  machine learning for restaurant business analytics. Internship
  worked specifically in video processing and feature extraction, such
as object detection and human pose estimation. I used Python, Pytorch,
C, and various freely available pretrained nets, as well as Amazon S3
and SageMaker.}
%\cvitem{2018}{Math Instructor, All-Girls Math League funded by the
%  Dragon Kim Foundation. I taught introductory combinatorics.}

\section{Awards}
\cvitem{2020}{NSERC USRA recipient in the Genov lab at UofT.}
\cvitem{2019}{ESROP-global recipient, placed in Thailand in
  association with King Mongkut's University of Technology Thonburi
  (KMUTT).}
\cvitem{2018-2019}{Dean's Honor List, Fall 2018, Winter 2019}
\cvitem{2018--2019}{University of Toronto Scholar}
%\cvitem{2017-2018}{California State Science Olympiad:  1st:
%  Hovercraft; 2nd: Fermi Questions; 3rd: Remote Sensing, 5th: Electric Vehicle}

\section{Extracurriculars}
\cvitem{2019--2020}{Vice President of Data Science Toronto Club, managing
  finances, helping organize club events, recruiting, and coming up
  with interesting data science problems for club members.}

\section{Computer skills}
\cvitem{Advanced}{Python, Pytorch}
\cvitem{}{Arduino C, various Arduino boards, developed PID drone flight controller.}
\cvitem{Intermediate}{C, Java, Verilog, Matlab, Fusion360}
\cvitem{}{\LaTeX}
\cvitem{Experience with:}{Vivado Design Suite}

%\section{Maker Skills}
%\cvlistitem{Proficient in soldering and various hand tools.}
%\cvlistitem{Experience with basic machining (lathing and milling) - Completed George Brown Basic Machining Course.}

\section{Links}
\cvitem{LinkedIn}{www.linkedin.com/in/louis-primeau}
\cvitem{GitHub}{https://github.com/louisprimeau}

%\section{Hobbies and Interests}
%\cvitem{2008-}{Violin, played since childhood, participated in chamber
%  music, orchestra}
%\cvitem{2014-}{Reading and Writing, In particular, plays and drama.}
% Publications from a BibTeX file without multibib
%  for numerical labels: \renewcommand{\bibliographyitemlabel}{\@biblabel{\arabic{enumiv}}}% CONSIDER MERGING WITH PREAMBLE PART
%  to redefine the heading string ("Publications"): \renewcommand{\refname}{Articles}
%\nocite{*}
%\bibliographystyle{plain}

% Publications from a BibTeX file using the multibib package
%\section{Publications}
%\nocitebook{book1,book2}
%\bibliographystylebook{plain}
%\bibliographybook{publications}                   % 'publications' is the name of a BibTeX file
%\nocitemisc{misc1,misc2,misc3}
%\bibliographystylemisc{plain}
%\bibliographymisc{publications}                   % 'publications' is the name of a BibTeX file
\end{document}


%% end of file `template.tex'.
