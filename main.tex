%% start of file `template.tex'.
%% Copyright 2006-2013 Xavier Danaux (xdanaux@gmail.com).
%
% This work may be distributed and/or modified under the
% conditions of the LaTeX Project Public License version 1.3c,
% available at http://www.latex-project.org/lppl/.


\documentclass[11pt,a4paper,sans]{moderncv}        % possible options include font size ('10pt', '11pt' and '12pt'), paper size ('a4paper', 'letterpaper', 'a5paper', 'legalpaper', 'executivepaper' and 'landscape') and font family ('sans' and 'roman')

% moderncv themes
\moderncvstyle{classic}                             % style options are 'casual' (default), 'classic', 'oldstyle' and 'banking'
\moderncvcolor{green}                               % color options 'blue' (default), 'orange', 'green', 'red', 'purple', 'grey' and 'black'
%\renewcommand{\familydefault}{\sfdefault}         % to set the default font; use '\sfdefault' for the default sans serif font, '\rmdefault' for the default roman one, or any tex font name
%\nopagenumbers{}                                  % uncomment to suppress automatic page numbering for CVs longer than one page

% character encoding
\usepackage[utf8]{inputenc}                       % if you are not using xelatex ou lualatex, replace by the encoding you are using
%\usepackage{CJKutf8}                              % if you need to use CJK to typeset your resume in Chinese, Japanese or Korean
\usepackage[document]{ragged2e}

% adjust the page margins
\usepackage[scale=0.75]{geometry}
%\setlength{\hintscolumnwidth}{3cm}                % if you want to change the width of the column with the dates
%\setlength{\makecvtitlenamewidth}{10cm}           % for the 'classic' style, if you want to force the width allocated to your name and avoid line breaks. be careful though, the length is normally calculated to avoid any overlap with your personal info; use this at your own typographical risks...

% personal data
\name{Louis}{Primeau}
%\title{Resumé title}                               % optional, remove / comment the line if not wanted
\address{89 Chestnut St}{M5G 1R1 Toronto}{Canada Room 2040}% optional, remove / comment the line if not wanted; the "postcode city" and and "country" arguments can be omitted or provided empty
\phone[mobile]{+1~(949)~572~7975}                   % optional, remove / comment the line if not wanted
%\phone[fixed]{+2~(345)~678~901}                    % optional, remove / comment the line if not wanted
%\phone[fax]{+3~(456)~789~012}                      % optional, remove / comment the line if not wanted
\email{louis.primeau@mail.utoronto.ca}                               % optional, remove / comment the line if not wanted
%\homepage{www.johndoe.com}                         % optional, remove / comment the line if not wanted
%\extrainfo{additional information}                 % optional, remove / comment the line if not wanted
%\photo[64pt][0.4pt]{picture}                       % optional, remove / comment the line if not wanted; '64pt' is the height the picture must be resized to, 0.4pt is the thickness of the frame around it (put it to 0pt for no frame) and 'picture' is the name of the picture file
%\quote{Some quote}                                 % optional, remove / comment the line if not wanted

% to show numerical labels in the bibliography (default is to show no labels); only useful if you make citations in your resume
%\makeatletter
%\renewcommand*{\bibliographyitemlabel}{\@biblabel{\arabic{enumiv}}}
%\makeatother
%\renewcommand*{\bibliographyitemlabel}{[\arabic{enumiv}]}% CONSIDER REPLACING THE ABOVE BY THIS

% bibliography with mutiple entries
%\usepackage{multibib}
%\newcites{book,misc}{{Books},{Others}}
%----------------------------------------------------------------------------------
%            content
%----------------------------------------------------------------------------------
\begin{document}
%\begin{CJK*}{UTF8}{gbsn}                          % to typeset your resume in Chinese using CJK
%-----       resume       ---------------------------------------------------------
\makecvtitle
\section{Career Goals}
\cvitem{}{Aiming to pursue graduate school in Physics.}
\section{Nationality}
\cvitem{Citizenship:}{US, Canada}
\section{Education}
\cventry{2018--}{Undergraduate}{University of
  Toronto}{Toronto, ON CA}{\textit{-}}{Major: Engineering Science,
  cGPA: 3.94}
\cventry{2014--2018}{High School}{Orange County School of the
  Arts}{Santa Ana, CA USA}{\textit{-}}{Creative Writing Conservatory}
\section{Work Experience}
\cvitem{2019}{Data Science Intern, EATLAB, a startup working in
  machine learning for restaurant business analytics. Internship
  worked specifically in video processing and feature extraction, such
as object detection and human pose estimation. }
\cvitem{2018}{Math Instructor, All-Girls Math League funded by the
  Dragon Kim Foundation.}
\section{Computer skills}
\cvitem{Advanced}{Python, Pandas}
\cvitem{}{Arduino C, used Nanos, Unos, 101s,
  Dues, several specific sensor libraries, developed PID controller.}
\cvitem{Intermediate}{Stan, Pytorch, C}
\cvitem{}{Java, Object oriented functionality}
\cvitem{}{Latex, Bibtex, Overleaf}
\cvitem{Beginner}{Git, Matlab, Keras}

\section{Maker Skills}
\cvlistitem{Proficient in soldering and various hand tools.}
\cvlistitem{Experience with 3d printing and 3d modelling.}
\cvlistitem{Experience with basic machining (lathing and milling) - Completed George Brown Basic Machining Course.}

\section{Awards}
\cvitem{2018-2019}{University of Toronto Scholar}
\cvitem{2018}{California State Science Olympiad: 3rd Place Remote
  Sensing}
\cvitem{2017}{1st Place: Hovercraft; 2nd Place: Fermi Questions; 5th
  Place: Electric Vehicle}
\cvitem{2015-2016}{Science Olympiad Regional Awards}

%\section{Hobbies and Interests}
%\cvitem{2008-}{Violin, played since childhood, participated in chamber
%  music, orchestra}
%\cvitem{2014-}{Reading and Writing, In particular, plays and drama.}
% Publications from a BibTeX file without multibib
%  for numerical labels: \renewcommand{\bibliographyitemlabel}{\@biblabel{\arabic{enumiv}}}% CONSIDER MERGING WITH PREAMBLE PART
%  to redefine the heading string ("Publications"): \renewcommand{\refname}{Articles}
%\nocite{*}
%\bibliographystyle{plain}

% Publications from a BibTeX file using the multibib package
%\section{Publications}
%\nocitebook{book1,book2}
%\bibliographystylebook{plain}
%\bibliographybook{publications}                   % 'publications' is the name of a BibTeX file
%\nocitemisc{misc1,misc2,misc3}
%\bibliographystylemisc{plain}
%\bibliographymisc{publications}                   % 'publications' is the name of a BibTeX file
\end{document}


%% end of file `template.tex'.
